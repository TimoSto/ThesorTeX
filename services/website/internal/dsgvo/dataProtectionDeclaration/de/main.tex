\documentclass[12pt]{article}

% To filter warning messages
\usepackage{silence}
% Disable all warnings issued by latex starting with "You have requested package" (caused by local imports)
\WarningFilter{latex}{You have requested package}

% General settings

% Paper spacings
\usepackage[a4paper,left=3cm,right=3cm,top=3cm,bottom=3.5cm,headsep=12pt]{geometry}%Bottom extra 0.5cm für Footer

% german language package (for Umlauts)
\usepackage[german,ngerman]{babel}

% For synmbols like \degree
\usepackage{gensymb}

% Times New Roman
\usepackage{mathptmx}

% To include a pdf file (e.g. the title page)
\usepackage{pdfpages}

% 1.5 line spacing
\usepackage[onehalfspacing]{setspace}

% Styling of the headlines, see styPackages/headlines.sty
\usepackage[numeric]{styPackages/headings}

% Style of the table of contents, see styPAckages/tableOfContents.sty
\usepackage[numeric]{styPackages/table_of_contents}

% Implementation of the list of abbreviations, see styPackages/abbreviations.sty
\usepackage{styPackages/abbreviations}

% Style (font size and spacing) of the footer, see styPackages/footer.sty
\usepackage{styPackages/footer}

% Bibliography and quotes, see styPackages/bibliography.sty
% This is an alternative to BibLaTeX
\usepackage{styPackages/bibliography}

% Change the title of the bibliography
\renewcommand{\bibliographytitle}{My Bibliography}

% Style and contents of the header and footer, see styPackages/header_footer.sty
% If you dont want to include this package, the commands \frontmatter and \mainmatter must be removed too
\usepackage{styPackages/header_footer}

% Style for code snippets, see codes.sty
% If you dont have any code snippets in your document, this package can be removed
\usepackage{styPackages/codes}
% Special code style for html, see styPackages/html.sty
\usepackage{styPackages/html}

% Style and contents for the appendix and the list of appendix, see styPAckages/appendix.sty
\usepackage{styPackages/appendix}

% Extend the hyphenation with unknown cases
\usepackage[T1]{fontenc}
\hyphenation{öf-fent-lich-en}

% DEBUGGING (Shows the boxes of header, footer and content)
%\usepackage{showframe}

\usepackage{makecell}
\usepackage{placeins}

% Change the style of quotation marks
\usepackage{csquotes}
\MakeOuterQuote{"}

% To improve the placing of figures and tables
\usepackage{float}

% TODO: can be deleted?
\usepackage{verbatimbox}

% For images
\usepackage{graphicx}

% Change the naming of figures and tables to the german abbreviations
\addto\captionsngerman{\renewcommand{\figurename}{Abb.}}
\addto\captionsngerman{\renewcommand{\tablename}{Tab.}}

% No indent in the list of figures
\setlength{\cftfigindent}{0 em}

% Center the heading of the list of figures
\renewcommand{\cftloftitlefont}{\hspace*{\fill}\LARGE}
\renewcommand{\cftafterloftitle}{\hspace*{\fill}}

% No indent in the list of tables
\setlength{\cfttabindent}{0 em}

% Center the heading of the list of tables
\renewcommand{\cftlottitlefont}{\hspace*{\fill}\LARGE}
\renewcommand{\cftafterlottitle}{\hspace*{\fill}}

% To properly include images and tables floating in text
\usepackage{wrapfig}
\usepackage{etoolbox}
\BeforeBeginEnvironment{wraptable}{\setlength{\intextsep}{1pt}}
\usepackage[justification=centering]{caption}

% Helvetica font (similar to Arial)
% WARNING: If you use this font, you cannot have multiline values inside the header. This is already a known issue (https://github.com/TimoSto/ThesorTeX/issues/128)
\usepackage{helvet}
%\renewcommand{\familydefault}{\sfdefault}

% To create bookmarks for the chapters in the pdf
\newcounter{dummy}
\usepackage[hidelinks,bookmarks=true]{hyperref}
\usepackage{bookmark}

% Improve linebreaks
\usepackage[copyfonts,activate={true,nocompatibility},final,tracking=true,kerning=true,spacing=true,factor=1000,stretch=10,shrink=10]{microtype}
\SetExpansion
[ context = sloppy,
  stretch = 30,
  shrink = 60,
  step = 5 ]
{ encoding = {OT1,T1,TS1} }
{ }

% Styles of urls
\urlstyle{same}
\renewcommand\UrlFont\itshape
\usepackage{xurl}

\partOnSamePage

\begin{document}

\renewcommand{\mytitle}{Datenschutzerklärung}% Title to show in the header or footer
\renewcommand{\myauthor}{ThesorTeX}% Name to show in the header or footer

% Style of header and footer for table of contents, bibliography, ...
% This style is activated using the command \frontmatter
% first param: top left
% second param: top right
% third param: bottom left
% forth param: bottom right
% the \plaintitle value is set manually (see below)
% the \thepage value is set automatically by LaTeX
\setPlainPageStyle{\mytitle}{\nouppercase\plaintitle}{\myauthor}{\thepage}

% Style of header and footer for the text part of your document
% This style is activated using the command \mainmatter
% the \parttitle value refers to the current top level chapter on a page (\part)
\setMainPageStyle{\mytitle}{\nouppercase\parttitle}{\myauthor}{\thepage}

% Here you can include one or more pages from a pdf file, e.g. as the title page
%\includepdf[pages={1-}]{titlepage.pdf}

% Activate the frontmatter style for the header and footer
\frontmatter

\pagenumbering{arabic}

% Change the title of the toc
\renewcommand{\contentsname}{Inhaltsverzeichnis}
% Change the title in the header
\renewcommand{\plaintitle}{\contentsname}
% This construct removes some spacing between the dotted line and the page number
{\def\makebox[#1][#2]#3{#3}%
    \tableofcontents
}

% Add vertical spacing to the table of contents
\addtocontents{toc}{\vspace{24pt}}

\clearpage

% Switch the style of the header and footer
\mainmatter
% Use arabic page numbering


% Start Content
\part{Präamble}
Mit der folgenden Datenschutzerklärung möchten wir Sie darüber aufklären, welche Arten Ihrer personenbezogenen Daten (nachfolgend auch kurz als "Daten" bezeichnet) wir zu welchen Zwecken und in welchem Umfang verarbeiten. Die Datenschutzerklärung gilt für alle von uns durchgeführten Verarbeitungen personenbezogener Daten, sowohl im Rahmen der Erbringung unserer Leistungen als auch insbesondere auf unseren Webseiten, in mobilen Applikationen sowie innerhalb externer Onlinepräsenzen, wie z. B. unserer Social-Media-Profile (nachfolgend zusammenfassend bezeichnet als "Onlineangebot").\\
Die verwendeten Begriffe sind nicht geschlechtsspezifisch.\\
Stand: 28.10.2023

\part{Verantwortlicher}
Timo Stovermann
Bree 8
46354 Südlohn
thesortex.contact@gmail.com

\part{Übersicht der Verarbeitung}
Die nachfolgende Übersicht fasst die Arten der verarbeiteten Daten und die Zwecke ihrer Verarbeitung zusammen und verweist auf die betroffenen Personen.
\section*{Arten der verarbeiteten Daten}
\begin{itemize}
\item Nutzungsdaten
\item Meta-, Kommunikations- und Verfahrensdaten
\end{itemize}
\section*{Kategorien betroffener Nutzer}
\begin{itemize}
\item Besucher der Online-Angebote
\end{itemize}
\section*{Zwecke der Verarbeitung}
\begin{itemize}
\item Sicherheitsmaßnahmen
\item Bereitstellung unseres Onlineangebotes und Nutzerfreundlichkeit
\item Informationstechnische Infrastruktur
\end{itemize}

\part{Rechtsgrundlage}
Im Folgenden erhalten Sie eine Übersicht der Rechtsgrundlagen der DSGVO, auf deren Basis wir personenbezogene Daten verarbeiten. Bitte nehmen Sie zur Kenntnis, dass neben den Regelungen der DSGVO nationale Datenschutzvorgaben in Ihrem bzw. unserem Wohn- oder Sitzland gelten können. Sollten ferner im Einzelfall speziellere Rechtsgrundlagen maßgeblich sein, teilen wir Ihnen diese in der Datenschutzerklärung mit.\\
\textbf{Berechtigte Interessen (Art. 6 Abs. 1 S. 1 lit. f) DSGVO)}: Die Verarbeitung ist zur Wahrung der berechtigten Interessen des Verantwortlichen oder eines Dritten erforderlich, sofern nicht die Interessen oder Grundrechte und Grundfreiheiten der betroffenen Person, die den Schutz personenbezogener Daten erfordern, überwiegen.\\
\textbf{Nationale Datenschutzregelungen in Deutschland}: Zusätzlich zu den Datenschutzregelungen der DSGVO gelten nationale Regelungen zum Datenschutz in Deutschland. Hierzu gehört insbesondere das Gesetz zum Schutz vor Missbrauch personenbezogener Daten bei der Datenverarbeitung (Bundesdatenschutzgesetz – BDSG). Das BDSG enthält insbesondere Spezialregelungen zum Recht auf Auskunft, zum Recht auf Löschung, zum Widerspruchsrecht, zur Verarbeitung besonderer Kategorien personenbezogener Daten, zur Verarbeitung für andere Zwecke und zur Übermittlung sowie automatisierten Entscheidungsfindung im Einzelfall einschließlich Profiling. Ferner können Landesdatenschutzgesetze der einzelnen Bundesländer zur Anwendung gelangen.

\part{Sicherheitsmaßnahmen}
Wir treffen nach Maßgabe der gesetzlichen Vorgaben unter Berücksichtigung des Stands der Technik, der Implementierungskosten und der Art, des Umfangs, der Umstände und der Zwecke der Verarbeitung sowie der unterschiedlichen Eintrittswahrscheinlichkeiten und des Ausmaßes der Bedrohung der Rechte und Freiheiten natürlicher Personen geeignete technische und organisatorische Maßnahmen, um ein dem Risiko angemessenes Schutzniveau zu gewährleisten.\\
Zu den Maßnahmen gehören insbesondere die Sicherung der Vertraulichkeit, Integrität und Verfügbarkeit von Daten durch Kontrolle des physischen und elektronischen Zugangs zu den Daten als auch des sie betreffenden Zugriffs, der Eingabe, der Weitergabe, der Sicherung der Verfügbarkeit und ihrer Trennung. Des Weiteren haben wir Verfahren eingerichtet, die eine Wahrnehmung von Betroffenenrechten, die Löschung von Daten und Reaktionen auf die Gefährdung der Daten gewährleisten. Ferner berücksichtigen wir den Schutz personenbezogener Daten bereits bei der Entwicklung bzw. Auswahl von Hardware, Software sowie Verfahren entsprechend dem Prinzip des Datenschutzes, durch Technikgestaltung und durch datenschutzfreundliche Voreinstellungen.\\
Um die Daten der Benutzer, die über unsere Online-Dienste übertragen werden, zu schützen, verwenden wir TLS/SSL-Verschlüsselung. Secure Sockets Layer (SSL) ist die Standardtechnologie zur Sicherung von Internetverbindungen durch Verschlüsselung der zwischen einer Website oder App und einem Browser (oder zwischen zwei Servern) übertragenen Daten. Transport Layer Security (TLS) ist eine aktualisierte und sicherere Version von SSL. Hyper Text Transfer Protocol Secure (HTTPS) wird in der URL angezeigt, wenn eine Website durch ein SSL/TLS-Zertifikat gesichert ist.

\part{Nutzung von Content-Delivery-Networks}
Wir nutzen das Content Delivery Network (CDN) Amazon CloudFront von Amazon Web Services EMEA SARL, 38 avenue John F. Kennedy, L-1855 Luxembourg (AWS), um die Sicherheit und die Auslieferungsgeschwindigkeit unserer Website zu erhöhen. Dies entspricht unserem berechtigten Interesse (Art. 6 Abs. 1 lit. f DSGVO). Ein CDN ist ein Netzwerk aus weltweit verteilten Servern, das in der Lage ist, optimiert Inhalte an den Websitenutzer auszuliefern. Für diesen Zweck können personenbezogene Daten in Server-Logfiles von AWS verarbeitet werden. Bitte vergleichen Sie die Ausführungen unter „Hosting“. Zusätzlich bewahren wir anonymisierte Logfiles auf, um die Stabilität und Sicherheit unserer Website zu gewährleisten.\\
AWS ist Empfänger Ihrer personenbezogenen Daten und als Auftragsverarbeiter für uns tätig. Die entspricht unserem berechtigten Interesse im Sinne des Art. 6 Abs. 1 S. 1 lit. f DSGVO, selbst kein Content Delivery Network zu betreiben.\\
Sie haben das Recht der Verarbeitung zu widersprechen. Ob der Widerspruch erfolgreich ist, ist im Rahmen einer Interessenabwägung zu ermitteln. Die Verarbeitung der unter diesem Abschnitt angegebenen Daten ist weder gesetzlich noch vertraglich vorgeschrieben. Die Funktionsfähigkeit der Website ist ohne die Verarbeitung nicht gewährleistet. Ihre personenbezogenen Daten werden von AWS so lange gespeichert, wie es für die beschriebenen Zwecke erforderlich ist.\\
Weitere Informationen zu Widerspruchs- und Beseitigungsmöglichkeiten gegenüber AWS finden Sie unter: \url{https://d1.awsstatic.com/legal/privacypolicy/AWS_Privacy_Notice__German_Translation.pdf} AWS hat Compliance-Maßnahmen für internationale Datenübermittlungen umgesetzt. Diese gelten für alle weltweiten Aktivitäten, bei denen AWS personenbezogene Daten von natürlichen Personen in der EU verarbeitet. Diese Maßnahmen basieren auf den EU-Standardvertragsklauseln (SCCs). Weitere Informationen finden Sie unter: \url{https://d1.awsstatic.com/legal/aws-gdpr/AWS_GDPR_DPA.pdf}

\part{Löschung von Daten}
Die von uns verarbeiteten Daten werden nach Maßgabe der gesetzlichen Vorgaben gelöscht, sobald deren zur Verarbeitung erlaubten Einwilligungen widerrufen werden oder sonstige Erlaubnisse entfallen (z. B. wenn der Zweck der Verarbeitung dieser Daten entfallen ist oder sie für den Zweck nicht erforderlich sind). Sofern die Daten nicht gelöscht werden, weil sie für andere und gesetzlich zulässige Zwecke erforderlich sind, wird deren Verarbeitung auf diese Zwecke beschränkt. D. h., die Daten werden gesperrt und nicht für andere Zwecke verarbeitet. Das gilt z. B. für Daten, die aus handels- oder steuerrechtlichen Gründen aufbewahrt werden müssen oder deren Speicherung zur Geltendmachung, Ausübung oder Verteidigung von Rechtsansprüchen oder zum Schutz der Rechte einer anderen natürlichen oder juristischen Person erforderlich ist. Unsere Datenschutzhinweise können ferner weitere Angaben zu der Aufbewahrung und Löschung von Daten beinhalten, die für die jeweiligen Verarbeitungen vorrangig gelten.

\part{Rechte der betroffenen Person}
Rechte der betroffenen Personen aus der DSGVO: Ihnen stehen als Betroffene nach der DSGVO verschiedene Rechte zu, die sich insbesondere aus Art. 15 bis 21 DSGVO ergeben:

\section{Widerspruchsrecht}
Sie haben das Recht, aus Gründen, die sich aus Ihrer besonderen Situation ergeben, jederzeit gegen die Verarbeitung der Sie betreffenden personenbezogenen Daten, die aufgrund von Art. 6 Abs. 1 lit. e oder f DSGVO erfolgt, Widerspruch einzulegen; dies gilt auch für ein auf diese Bestimmungen gestütztes Profiling. Werden die Sie betreffenden personenbezogenen Daten verarbeitet, um Direktwerbung zu betreiben, haben Sie das Recht, jederzeit Widerspruch gegen die Verarbeitung der Sie betreffenden personenbezogenen Daten zum Zwecke derartiger Werbung einzulegen; dies gilt auch für das Profiling, soweit es mit solcher Direktwerbung in Verbindung steht.

\section{Widerrufsrecht bei Einwilligungen}
Sie haben das Recht, erteilte Einwilligungen jederzeit zu widerrufen.

\section{Auskunftsrecht}
Sie haben das Recht, eine Bestätigung darüber zu verlangen, ob betreffende Daten verarbeitet werden und auf Auskunft über diese Daten sowie auf weitere Informationen und Kopie der Daten entsprechend den gesetzlichen Vorgaben.

\section{Recht auf Berichtigung}
Sie haben entsprechend den gesetzlichen Vorgaben das Recht, die Vervollständigung der Sie betreffenden Daten oder die Berichtigung der Sie betreffenden unrichtigen Daten zu verlangen.

\section{Recht auf Löschung}
Sie haben nach Maßgabe der gesetzlichen Vorgaben das Recht, zu verlangen, dass Sie betreffende Daten unverzüglich gelöscht werden.

\part{Bereitstellung des Onlineangebotes und Webhosting}
Wir verarbeiten die Daten der Nutzer, um ihnen unsere Online-Dienste zur Verfügung stellen zu können. Zu diesem Zweck verarbeiten wir die IP-Adresse des Nutzers, die notwendig ist, um die Inhalte und Funktionen unserer Online-Dienste an den Browser oder das Endgerät der Nutzer zu übermitteln.\\
\section{Verarbeitete Datenarten}
\begin{itemize}
\item Nutzungsdaten (z. B. besuchte Webseiten, Interesse an Inhalten, Zugriffszeiten)
\item Meta-, Kommunikations- und Verfahrensdaten (z. .B. IP-Adressen, Zeitangaben, Identifikationsnummern, Einwilligungsstatus).
\end{itemize}
\section{Betroffene Personen}
\begin{itemize}
\item Nutzer (Webseitenbesuchern)
\end{itemize}
\section{Zwecke der Verarbeitung}
Bereitstellung unseres Onlineangebotes und Nutzerfreundlichkeit; Informationstechnische Infrastruktur (Betrieb und Bereitstellung von Informationssystemen und technischen Geräten (Computer, Server etc.).). Sicherheitsmaßnahmen.

\part{Weitere Hinweise zu Verarbeitungsprozessen, Verfahren und Diensten}
\section{Bereitstellung Onlineangebot auf gemietetem Speicherplatz}
Für die Bereitstellung unseres Onlineangebotes nutzen wir Speicherplatz, Rechenkapazität und Software, die wir von einem entsprechenden Serveranbieter (auch "Webhoster" genannt) mieten oder anderweitig beziehen; Rechtsgrundlagen: Berechtigte Interessen (Art. 6 Abs. 1 S. 1 lit. f) DSGVO).
\section{Erhebung von Zugriffsdaten und Logfiles}
Der Zugriff auf unser Onlineangebot wird in Form von so genannten "Server-Logfiles" protokolliert. Zu den Serverlogfiles können die Adresse und Name der abgerufenen Webseiten und Dateien, Datum und Uhrzeit des Abrufs, übertragene Datenmengen, Meldung über erfolgreichen Abruf, Browsertyp nebst Version, das Betriebssystem des Nutzers, Referrer URL (die zuvor besuchte Seite) und im Regelfall IP-Adressen und der anfragende Provider gehören. Die Serverlogfiles können zum einen zu Zwecken der Sicherheit eingesetzt werden, z. B., um eine Überlastung der Server zu vermeiden (insbesondere im Fall von missbräuchlichen Angriffen, sogenannten DDoS-Attacken) und zum anderen, um die Auslastung der Server und ihre Stabilität sicherzustellen; Rechtsgrundlagen: Berechtigte Interessen (Art. 6 Abs. 1 S. 1 lit. f) DSGVO).
\section{Content-Delivery-Network}
Wir setzen ein "Content-Delivery-Network" (CDN) ein. Ein CDN ist ein Dienst, mit dessen Hilfe Inhalte eines Onlineangebotes, insbesondere große Mediendateien, wie Grafiken oder Programm-Skripte, mit Hilfe regional verteilter und über das Internet verbundener Server schneller und sicherer ausgeliefert werden können; Rechtsgrundlagen: Berechtigte Interessen (Art. 6 Abs. 1 S. 1 lit. f) DSGVO).

\part{Änderung und Aktualisierung der Datenschutzerklärung}
Wir bitten Sie, sich regelmäßig über den Inhalt unserer Datenschutzerklärung zu informieren. Wir passen die Datenschutzerklärung an, sobald die Änderungen der von uns durchgeführten Datenverarbeitungen dies erforderlich machen. Wir informieren Sie, sobald durch die Änderungen eine Mitwirkungshandlung Ihrerseits (z. B. Einwilligung) oder eine sonstige individuelle Benachrichtigung erforderlich wird.

\part{Begriffsdefinitionen}
In diesem Abschnitt erhalten Sie eine Übersicht über die in dieser Datenschutzerklärung verwendeten Begrifflichkeiten. Soweit die Begrifflichkeiten gesetzlich definiert sind, gelten deren gesetzliche Definitionen. Die nachfolgenden Erläuterungen sollen dagegen vor allem dem Verständnis dienen.
\section{Personenbezogene Daten}
"Personenbezogene Daten" sind alle Informationen, die sich auf eine identifizierte oder identifizierbare natürliche Person (im Folgenden "betroffene Person") beziehen; als identifizierbar wird eine natürliche Person angesehen, die direkt oder indirekt, insbesondere mittels Zuordnung zu einer Kennung wie einem Namen, zu einer Kennnummer, zu Standortdaten, zu einer Online-Kennung (z. B. Cookie) oder zu einem oder mehreren besonderen Merkmalen identifiziert werden kann, die Ausdruck der physischen, physiologischen, genetischen, psychischen, wirtschaftlichen, kulturellen oder sozialen Identität dieser natürlichen Person sind.
\section{Verarbeitung}
"Verarbeitung" ist jeder mit oder ohne Hilfe automatisierter Verfahren ausgeführte Vorgang oder jede solche Vorgangsreihe im Zusammenhang mit personenbezogenen Daten. Der Begriff reicht weit und umfasst praktisch jeden Umgang mit Daten, sei es das Erheben, das Auswerten, das Speichern, das Übermitteln oder das Löschen.

\end{document}